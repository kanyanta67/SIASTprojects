\documentclass{ximera}
%% handout
%% nohints
%% space
%% newpage
%% numbers

%% You can put user macros here
%% However, you cannot make new environments

%\graphicspath{{./}{firstExample/}}

\usepackage{latexsym,amsfonts}
\usepackage{amssymb}
\usepackage{amsmath}
\usepackage{ifsym}
\usepackage{colortbl,pifont}
%%
\usepackage{tabularx}
\usepackage{xtab}
\usepackage{subfig}

\usepackage{etex}
\usepackage{fixltx2e}

\pgfplotsset{compat=1.10}

\usepackage{fancybox}
\usepackage{hyperref}
\usepackage{color}
\usepackage{amsbsy} %% for good bolded math!!

%%
%%%%%%%%%%%%%%%%%%%%%%%%%%%%%%%%%
\usepackage{graphicx,fancybox,multicol}

%%%%%%%%%%%%%%%%%%%%%%%%%%%%%%%
%\usepackage{enumitem}
\usepackage{xcolor}
\usepackage{subfig}
\usepackage{caption}
\usepackage{tabularx}
\usepackage{xtab}
\usepackage{boxedminipage}
\usepackage{empheq}

 \definecolor{brown}{rgb}{.65, .16, .16}
 \definecolor{lightblue}{rgb}{.68, .85, .9}
 \definecolor{palegreen}{rgb}{.6, .98, .6}
 \definecolor{pink}{rgb}{1, .75, .8}
 \definecolor{wheat}{rgb}{.96, .87, .7}
 \definecolor{newgray}{gray}{0.75}
 \definecolor{verylightgray}{gray}{0.95}
 \definecolor{shadecolor}{cmyk}{0,0,0.4,0}
 \definecolor{light-blue}{cmyk}{0.25,0,0,0}
%%%%%%%%%%%%%%%%%%%%%%%%%%%%%%%%%%%%%%%%

\usepackage{xkeyval}
\usepackage{pstricks,pst-math,pst-func}
\usepackage{pstricks-add}
 %% we can turn off input when making a master document

\prerequisites{none}
\outcomes{ximeraLatex}

\title{Physics SHM problems }

\begin{document}
\begin{abstract}
In this activity we see some examples on the use of the environment to create physics questions on SHM.
\end{abstract}
\maketitle

\section{Answer these multiple choice questions}

\begin{interactive}[interactiveContent.js]
\begin{question}
A mass on a spring undergoes SHM. When the mass is at its maximum displacement from equilibrium, its instantaneous velocity
\begin{solution}
\begin{multiple-choice}
\choice{is maximum}
\choice{is less than maximum, but not zero}
\choice[correct]{is zero}
\choice{cannot be determined from the information given}
\end{multiple-choice}
\begin{hint}
where is the displacement zero?
\end{hint}
\end{solution}
\end{question}

\begin{question}
A mass on a spring undergoes SHM. When the mass passes through the equilibrium position, its instantaneous velocity
\begin{solution}
\begin{multiple-choice}
\choice[correct]{is maximum}
\choice{is less than maximum, but not zero}
\choice{is zero}
\choice{cannot be determined from the information given}
\end{multiple-choice}
\end{solution}
\end{question}

\begin{question}
A mass is attached to a vertical spring and bobs up and down between points $A$ and $B$. Where is the mass located when its kinetic
energy is a minimum?
\begin{solution}
\begin{multiple-choice}
\choice[correct]{at either $A$ or $B$}
\choice{midway between $A$ and $B$}
\choice{one-fourth of the way between $A$ and $B$}
\choice{none of the above}
\end{multiple-choice}
\end{solution}
\end{question}

\begin{question}
A mass on a spring undergoes SHM. When the mass is at maximum displacement from equilibrium, its instantaneous acceleration
\begin{solution}
\begin{multiple-choice}
\choice{is zero}
\choice[correct]{is a maximum}
\choice{is less than maximum, but not zero}
\choice{cannot be determined from the information given}
\end{multiple-choice}
\end{solution}
\end{question}

\begin{question}
A mass is attached to a vertical spring and bobs up and down between points $A$ and $B$. Where is the mass located when its potential
energy is a maximum?
\begin{solution}
\begin{multiple-choice}
\choice{midway between $A$ and $B$}
\choice{one-fourth of the way between $A$ and $B$}
\choice[correct]{at either $A$ or $B$}
\choice{none of the above}
\end{multiple-choice}
\end{solution}
\end{question}

\begin{question}
A mass oscillates on the end of a spring, both on Earth and on the Moon. Where is the period the greatest?
\begin{solution}
\begin{multiple-choice}
\choice{Earth}
\choice{the Moon}
\choice[correct]{same on both Earth and the Moon}
\choice{cannot be determined from the information given}
\end{multiple-choice}
\end{solution}
\end{question}

\begin{question}
When the mass of a simple pendulum is tripled, the time required for one complete vibration
\begin{solution}
\begin{multiple-choice}
\choice{increases by a factor of $3$}
\choice{decreases to one-third of its original value}
\choice{decreases to $1/\sqrt{3}$ of its original value}
\choice[correct]{does not change}
\end{multiple-choice}
\end{solution}
\end{question}

\begin{question}
 A mass undergoes SHM with amplitude of $4$ cm. The energy is $8.0$ J at this time. The mass is cut in half and
the system is again set if motion with amplitude $4.0$ cm. What is the energy of the system now?
\begin{solution}
\begin{multiple-choice}
\choice{$2.0$ J}
\choice{$4.0$ J}
\choice[correct]{$8.0 J$}
\choice{$16$ J}
\end{multiple-choice}
\end{solution}
\end{question}

\section*{Short answer questions}
\begin{question}
A $2.00$ kg pumpkin oscillates from a vertically hanging light spring once every $0.65$ s. Write down the equation giving the pumpkin's position $y$ ($+$ upward) as a function of time $t$, assuming it started
by being compressed $18$ cm from the equilibrium position (where $y=0$) and released.
\begin{solution}
Since we are compressing the pumpkin to $18$ cm from its equilibrium position, the general equation for SHM $y=A\cos\left(\dfrac{2\pi t}{T}\right)$ gives
us
$$
y=(0.18\;\text{m})\cos\left(\dfrac{2\pi t}{0.65\;\text{s}}\right),
$$
for the position function.
\answer{$y=0.18\cos\left(\dfrac{2\pi t}{0.65}\right)$}
\end{solution}
How long will it take to get into the equilibrium position for the first time?
\begin{solution}
The time to return back to the equilibrium position is one-quarter of a period.

$$
t=\dfrac{1}{4}T=\dfrac{1}{4}(0.65\;\text{s})=\boxed{0.16\;\text{s}}
$$

\answer{$0.16$ s}
\end{solution}
What will be the pumpkin's maximum speed?
\begin{solution}
Here we have

$$
v_{\text{max}}=\omega A=\dfrac{2\pi}{T}A=\dfrac{2\pi}{0.65\;\text{s}}(0.18\;\text{m})=\boxed{1.7\;\text{m/s}}
$$
\answer{$1.7$ m/s}
\end{solution}
What will be its maximum acceleration and where will that first be attained?
\begin{solution}
This is given by
\begin{align*}
a_{\text{max}}&=\omega^2 A=\left(\dfrac{{2\pi}}{T}\right)^2A\\
&=\dfrac{4\pi^2}{(0.65\;\text{s})^2}(0.18\;\text{m})\\
&=\boxed{17\;\text{m/s}^2}.
\end{align*}

The maximum acceleration is first attained at the point of release of the pumpkin.
\answer{$17$ m/s$^2$, at $t=0$}
\end{solution}
\end{question}
%%
\begin{question}
A $300$ g mass vibrates according to the equation $x=0.38\sin 6.50 t$, where $x$ is in meters and $t$ is in seconds. Determine the amplitude.
\begin{solution}
This is $A=x_{\text{max}}=\boxed{0.38\;\text{m}}$
\answer{$0.38$ m}
\end{solution}
the frequency
\begin{solution}
We can find this as follows
$$
\omega = 2\pi f=6.50\;s^{-1}\rightarrow f=\dfrac{6.50\;s^{-1}}{2\pi}=\boxed{1.03\;\text{Hz}}
$$
\answer{$1.03$ Hz}
\end{solution}
the period
\begin{solution}
Here
$$
T=\dfrac{1}{f}=\dfrac{1}{1.03\;\text{Hz}}=\boxed{0.967\;\text{s}}
$$
\answer{$0.967$ s}
\end{solution}
the total energy
\begin{solution}
This is
\begin{align*}
E_{\text{total}}&=\dfrac{1}{2}mv^2_{\text{max}}=\dfrac{1}{2}m(\omega A)^2\\
&=\dfrac{1}{2}(0.300\;\text{kg})\left[(6.50\;s^{-1})(0.38\;\text{m})\right]^2\\
&=0.9151\;\text{J}\approx\boxed{0.92\;\text{J}}
\end{align*}
\answer{$0.92$ J}
\end{solution}
The KE and PE when $x$ is $9.0$ cm.
\begin{solution}
The potential energy is given by
\begin{align*}
E_{\text{potential}}&=\dfrac{1}{2}kx^2=\dfrac{1}{2}m\omega^2 x^2=\dfrac{1}{2}(0.300\;\text{kg})(6.50\;s^{-1})^2(0.090\;\text{m})^2\\
&=0.0513\;\text{J}\approx \boxed{5.1\times 10^{-2}\;\text{J}}
\end{align*}
The kinetic energy is given by
\begin{align*}
E_{\text{kinetic}}&=E_{\text{total}}-E_{\text{potential}}=0.9151\;\text{J}-0.0513\;\text{J}\\
&=0.8638\;\text{J}\approx \boxed{0.86\;\text{J}}.
\end{align*}
\answer{$KE=0.86$ J, $PE=0.051$ J}
\end{solution}

%\begin{solution}
%\answer{}
%\end{solution}
\end{question}
\end{interactive}

\end{document}
